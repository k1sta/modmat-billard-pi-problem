%% AI INTERPRETATION:
%% This Markdown document is a structured, slide-based presentation intended to explain a physical and mathematical model that relates billiards and the digits of π.
%% I interpret top-level headings (##) as individual slide titles, and lower-level headings (###) or bullet points as the content within each slide.
%% The initial title slide includes a main title, subtitle, and author list.
%% LaTeX Beamer will be used with a clean, modern theme and code listings, equations, and placeholder images preserved.
%% Each image reference will be replaced with a placeholder box and caption, with comments to guide image replacement.

\documentclass{beamer}
\usetheme{metropolis}
\usepackage[utf8]{inputenc}
\usepackage[brazil]{babel}
\usepackage{amsmath, amssymb}
\usepackage{graphicx}
\usepackage{tikz}
\usepackage{caption}
\usepackage{listings}
\usepackage{color}

\definecolor{codegray}{gray}{0.9}
\lstset{
    backgroundcolor=\color{codegray},
    basicstyle=\ttfamily\small,
    breaklines=true,
    frame=single,
    language=Python
}

\title{Jogando Sinuca com $\pi$}
\subtitle{Como o número de colisões em um sistema mecânico computa os dígitos de $\pi$}
\author{Leonardo Lima Santos, Lucas Pimentel Alves da Costa, Pedro Kury Kitagawa}
\date{}

\begin{document}

\begin{frame}
  \titlepage
\end{frame}

\begin{frame}{Definição do Problema}
\textbf{Objetivo:} Calcular a quantidade de colisões em um sistema idealizado
\begin{itemize}
  \item \textbf{Sistema Físico:}
  \begin{itemize}
    \item Parede fixa em $x = 0$
    \item Duas bolas de massa $m$ (pequena) e $M$ (grande)
    \item Movimento unidimensional (eixo $X$)
    \item Bola $m$ em repouso entre a parede e $M$
    \item Bola $M$ com velocidade $V$ em direção a $m$
  \end{itemize}
\end{itemize}
\end{frame}

\begin{frame}{Variáveis e Suposições do Sistema}
\begin{itemize}
  \item \textbf{Variáveis fundamentais}
    \begin{itemize}
      \item Massas $m$ e $M$
      \item Razão entre massas $\frac{M}{m} = 100^N$
    \end{itemize}
  \item \textbf{Simplificações}
    \begin{itemize}
      \item Colisões perfeitamente elásticas
      \item Sem atrito ou resistência do ar
      \item Bolas como partículas adimensionais
    \end{itemize}
\end{itemize}
\end{frame}

\begin{frame}{O que realmente acontece nas colisões?}
\textbf{Análise do caso simples $m = M$}
\[
\begin{aligned}
u_0 & = 0 \\
v_0 & = V \\
u_1 & = \frac{(m - M)u_0 + 2Mv_0}{m + M} = V \\
v_1 & = \frac{(M - m)v_0 + 2mu_0}{m + M} = 0
\end{aligned}
\]
\begin{enumerate}
  \item $M$ bate em $m$: $M$ para, $m$ segue com $V$
  \item $m$ bate na parede: volta com $-V$
  \item $m$ bate em $M$: $m$ para, $M$ segue com $-V$
  \item $M$ segue indefinidamente
\end{enumerate}
\end{frame}

\begin{frame}{O espaço de configuração do sistema}
\begin{itemize}
  \item $x(t)$: posição de $m$, $y(t)$: posição de $M$
  \item $P(t) = (x(t), y(t))$
  \item Restrições: $0 \leq x(t) \leq y(t)$
  \item Colisões:
    \begin{itemize}
      \item Bola-Parede: $x(t) = 0$, reflexão no eixo $Y$
      \item Bola-Bola: $x(t) = y(t)$, reflexão não trivial na fronteira $x = y$
    \end{itemize}
\end{itemize}
\end{frame}

% Slides 6–10: Use consistent structure with image placeholders
\begin{frame}{Início do experimento}
$\vec{p} = (x(t),y(t)) = (u,v), \quad \vec{\dot p} = (\dot x(t),\dot y(t))$
\vspace{1em}

\includegraphics[width=\linewidth]{example-image}
\captionof*{figure}{\textit{Placeholder for image1-1.png. Replace with actual image.}}
\end{frame}

\begin{frame}{Primeira colisão entre $m$ e $M$}
$\vec{\dot p}(0) = (0,V) \vdash \vec{\dot p}(t_1) = (V,0)$

\includegraphics[width=\linewidth]{example-image}
\captionof*{figure}{\textit{Placeholder for image1-2.png. Replace with actual image.}}
\end{frame}

\begin{frame}{Colisão entre $m$ e a parede}
$\vec{\dot p}(t_1) = (V,0) \vdash \vec{\dot p}(t_2) = (-V,0)$

\includegraphics[width=\linewidth]{example-image}
\captionof*{figure}{\textit{Placeholder for image1-3.png. Replace with actual image.}}
\end{frame}

\begin{frame}{Segunda colisão entre $m$ e $M$}
$\vec{\dot p}(t_2) = (-V,0) \vdash \vec{\dot p}(t_3) = (0,-V)$

\includegraphics[width=\linewidth]{example-image}
\captionof*{figure}{\textit{Placeholder for image1-4.png. Replace with actual image.}}
\end{frame}

\begin{frame}{$M$ segue infinitamente, com $m$ parado}
$\vec{\dot p}(t_{\infty}) = (0,-V)$

\includegraphics[width=\linewidth]{example-image}
\captionof*{figure}{\textit{Placeholder for image1-5.png. Replace with actual image.}}
\end{frame}

\begin{frame}{O que acontece quando $m \neq M$?}
\begin{itemize}
  \item Reflexões não são mais ópticas
  \item Não sabemos se $P$ retorna (se as colisões param)
\end{itemize}

\includegraphics[width=\linewidth]{example-image}
\captionof*{figure}{\textit{Placeholder for image-2.png. Replace with actual image.}}
\end{frame}

\begin{frame}{O pulo do gato}
\textbf{A chave para simplificar o problema:}
\begin{itemize}
  \item Transformar a reflexão em $x = y$ em reflexão óptica
  \item Novas coordenadas:
  \[
  x' = \sqrt{m} \cdot x,\quad y' = \sqrt{M} \cdot y
  \]
  \[
  \vec{p} = \begin{pmatrix} x \\ y \end{pmatrix},\quad
  \vec{p'} = \begin{pmatrix} x' \\ y' \end{pmatrix} = T \cdot \vec{p}
  \]
  \[
  T = \begin{pmatrix} \sqrt{m} & 0 \\ 0 & \sqrt{M} \end{pmatrix}
  \]
\end{itemize}
\end{frame}

\begin{frame}{Novo espaço após aplicação de $T$}
\begin{itemize}
  \item O ângulo entre eixo $X$ e $x=y$ muda
\end{itemize}

\includegraphics[width=\linewidth]{example-image}
\captionof*{figure}{\textit{Placeholder for image-3.png. Replace with actual image.}}
\end{frame}

\begin{frame}{Porque aplicamos $T$?}
\begin{itemize}
  \item Queremos reflexões ópticas
  \item Após aplicar $T$, podemos refletir todo o espaço
  \item Velocidade transformada igual à posição transformada
  \item Vamos analisar:
    \begin{itemize}
      \item Reflexão na parede $x(t) = 0$
      \item Reflexão no eixo $Y = \sqrt{\frac{M}{m}}X$
    \end{itemize}
\end{itemize}
\end{frame}

\begin{frame}{Reflexão no eixo $Y$}
\begin{itemize}
  \item Após reflexão na parede: $\vec{\dot p'} = \begin{pmatrix} -\sqrt{m}u \\ \sqrt{M}v \end{pmatrix}$
  \item Ângulo antes e depois iguais $\Rightarrow$ reflexão óptica
\end{itemize}

\includegraphics[width=\linewidth]{example-image}
\captionof*{figure}{\textit{Placeholder for image-4.png. Replace with actual image.}}

\[
\blacksquare \; 1/2
\]
\end{frame}

\begin{frame}{Reflexão no eixo $Y = \sqrt{\frac{M}{m}}X$}
\begin{itemize}
  \item Conservação de momento e energia:
  \[
  \begin{cases}
  mu + Mv = K_1 \\
  mu^2 + Mv^2 = K_2
  \end{cases}
  \]
  \item Em coordenadas transformadas:
  \[
  \begin{cases}
  \vec{m} \cdot \vec{\dot p'} = K_1 \\
  |\vec{\dot p'}|^2 = K_2
  \end{cases}
  \]
  \item $\cos \varphi = \cos \psi \Rightarrow$ reflexão óptica
\end{itemize}

\[
\blacksquare \; 2/2
\]
\end{frame}

\begin{frame}{Número de Reflexões Ópticas}
\begin{itemize}
  \item Precisamos mostrar que o número de reflexões é finito
  \item \textbf{Lemma:}
    \begin{itemize}
      \item (a) Máximo número de reflexões num ângulo $\alpha$ é finito
      \item (b) Esse número é $\lceil \frac{\pi}{\alpha} \rceil$
      \item (c) Se raio inicial for paralelo a um lado, número é $\lceil \frac{\pi}{\alpha} \rceil - 1$
    \end{itemize}
\end{itemize}
\end{frame}

\begin{frame}{Prova do Lemma anterior}
\begin{itemize}
  \item Consideramos:
    \begin{itemize}
      \item Trajetória $\gamma$
      \item Ângulo $\alpha$
    \end{itemize}
  \item (Prova continua nas próximas slides — se necessário)
\end{itemize}
\end{frame}

\end{document}
